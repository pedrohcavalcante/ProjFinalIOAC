\documentclass[12pt, oneside,a4paper, brazil]{abntex2}
% \documentclass[article,11pt,oneside,a4paper,twocolumn]{abntex2}
\usepackage[T1]{fontenc}
%\usepackage{natbib}
\usepackage[utf8]{inputenc}
\usepackage{indentfirst}
\usepackage{url}
\usepackage{xcolor}
\definecolor{verde}{rgb}{0.25,0.5,0.35}
\definecolor{jpurple}{rgb}{0.5,0,0.35}
\usepackage{listings}
\lstset{
  language=C++,
  basicstyle=\ttfamily\small,
  keywordstyle=\color{jpurple}\bfseries,
  stringstyle=\color{red},
  commentstyle=\color{verde},
  morecomment=[s][\color{blue}]{/**}{*/},
  extendedchars=true,
  showspaces=false,
  showstringspaces=false,
  numbers=left,
  numberstyle=\tiny,
  breaklines=true,
  backgroundcolor=\color{cyan!10},
  breakautoindent=true,
  captionpos=b,
  xleftmargin=0pt,
  tabsize=4
}
\pagestyle{empty}
\usepackage{lmodern}
\usepackage{epigraph}
\usepackage{microtype}
% \usepackage[brazilian,hyperpageref]{backref} Coloca na referência a página que está a citação
\usepackage[alf]{abntex2cite}	
%\usepackage[pdftex]{color,graphicx}
\title{Relatório Final}
\instituicao{Universidade Federal do Rio Grande do Norte}
\local{Natal - RN}

\autor{Pedro Henrique Bezerra Cavalcante}
\date{\today}
\preambulo{Relatório final para a disciplina de Introdução à Organização e Arquitetura de Computadores}
\orientador{Mônica Magalhães Pereira}
\tipotrabalho{relatorio}
\setlrmarginsandblock{3cm}{3cm}{*}
\setulmarginsandblock{3cm}{3cm}{*}
\setlength{\parindent}{1.25cm} %Funcionou 
\SingleSpacing

\begin{document}
\selectlanguage{brazil}

\frenchspacing
\imprimircapa
\imprimirfolhaderosto

\tableofcontents
\chapter{Introdução}
\section{Daisy Chaining}
Daisy Chaining é uma topologia utilizada em vários aspectos, como por exemplo o árbitro de barramento ou topologia de rede. Ambos seguem a mesma lógica: o dispositivo solicita execução através do caminho destinado para esse fim (bus-request). Por estarem ligados em série, o árbitro responde à solicitação passando em cada periférico e vendo se ele foi quem solocitou. Caso não seja, passa para o próximo, até encontrar aquele que solicitou. 
\section{Prioridade}
A forma de árbitrariedade por prioridade é executada de acordo com a prioridade inerente ao dispositivo que solicitou execução. 
\section{Justiça}
Arbritrariedade por tempo de justiça é dada da forma que: cada dispositivo tem seu tempo de execução. Por justiça, o árbitro é implementado com um tempo fixo de execução. Caso o dispositivo leve mais tempo que o árbitro permite, ele é executado, interrompido e passa para o final da linha de execução. Dessa forma, 
\chapter{Solução e Implementação}
A solução encontrada para o resolvimento desse projeto foi encontrada, para Daisy Chaining, a execução dos periféricos de acordo com a entrada da solicitação, ou seja, pelo ordem que a requisição vai chegando ao barramento, ela vai ficar na fila e será executada conforme essa ordem. 

Para prioridade, foi realizada uma ordenação utilizando o algoritmo de Ordenação por Seleção, de acordo com a prioridade armazenada na estrutura. 

Foi desenvolvida uma lista encadeada com duas estruturas,mostradas à seguir: 
\begin{lstlisting}
typedef struct{
	int tipo;
	char dispositivo[50];
	int prioridade;
}Perif;

typedef struct no{
	Perif info;
	struct no* prox;
}Aux_Perif;
\end{lstlisting}

A estrutura Aux\_perif é formada pelo ponteiro para o próximo no e uma estrutura Perif. Essa última, é formada pelos dados do periférico, que é seu tipo (int), o nome do dispositivo (char) e a sua prioridade (int).


\begin{lstlisting}
/**
* Arquivo main.cpp
*/
#include <iostream>
#include <string.h>
#include <stdlib.h>

int main(){
	int solic;
	int cont = 0;
	int aux = 0, prior = 0;
	Aux_Perif* perifericos;
	perifericos = criarLista();
	std::cout << "Entre com o numero de solicitacoes: " << std::endl;
	std::cin >> solic;
	std::cout << "Perifericos:\n1. Impressora\n2. Mouse\n3. Teclado\n4. Scanner\n5. Roteador\n6. Disco Rigido" << std::endl;
	std::cout << "De acordo com os numeros e Perifericos acima, insira a ordem de entrada das solicitacoes e sua prioridade" << std::endl;
	while (cont < solic){
		Perif ordem;
		std::cin >> aux >> prior;
		switch (aux){
			case 1:
				//ordem.dispositivo = 'Impressora';
				strcpy(ordem.dispositivo, "Impressora");
				break;
			case 2: 
				strcpy(ordem.dispositivo , "Mouse") ;
				break;
			case 3: 
				strcpy(ordem.dispositivo , "Teclado");
				break;
			case 4:
				strcpy(ordem.dispositivo , "Scanner");
				break;
			case 5:
				strcpy(ordem.dispositivo , "Roteador");
				break;
			case 6: 
				strcpy(ordem.dispositivo , "Disco Rigido");
				break;
			default:
				std::cout << "Error" << std::endl;
				break;
		}
		
		ordem.tipo = aux;
		ordem.prioridade = prior;
		cont++;
		inserirFinal(&perifericos, ordem);
		//delete[] ordem;
		
	}
	std::cout << "Daisy Chaining utilizando metodo de ordem a chegada ou entrada." << std::endl;
	imprimirLista(&perifericos);
	std::cout << "METODO POR PRIORIDADE: " << std::endl;
	ordena_AuxPerif(&perifericos);
	imprimirLista(&perifericos);
	return 0;
}
/**
* Funcoes utilizadas
*/
typedef struct{
	int tipo;
	char dispositivo[50];
	int prioridade;
}Perif;

typedef struct no{
	Perif info;
	struct no* prox;
}Aux_Perif;

Aux_Perif* criarLista(){
	return NULL;
}
void selectionsort_AuxPerif(Aux_Perif **vetor, int tamanho) {
    int i, j;
    for (i = 0; i < tamanho - 1; i++) {
        int menor = i;
        for (j = i + 1; j < tamanho; j++) {
            if (vetor[menor]->info.prioridade > vetor[j]->info.prioridade) menor = j;
        }
        if (menor != i) {
            Aux_Perif *temp = vetor[menor];
            vetor[menor] = vetor[i];
            vetor[i] = temp;
        }
    }
}

void ordena_AuxPerif(Aux_Perif **perif) {
    // 1. Se a lista esta vazia, entao nem faz nada.
    if (perif == NULL) return;

    // 2. Descobre o tamanho da lista.
    int tamanho = 0;
    Aux_Perif *v;
    for (v = *perif; v != NULL; v = v->prox) {
        tamanho++;
    }

    // 3. Monta um vetor com os elementos da lista.
    Aux_Perif **vetor = (Aux_Perif **) malloc(sizeof(Aux_Perif *) * tamanho);
    int i = 0;
    for (v = *perif; v != NULL; i++) {
        vetor[i] = v;
        v = v->prox;
    }

    // 4. Ordena o vetor.
    selectionsort_AuxPerif(vetor, tamanho);

    // 5. Corrige os ponteiros de acordo com a nova ordenacao.
    for (i = 0; i < tamanho - 1; i++) {
        vetor[i]->prox = vetor[i + 1];
    }
    vetor[i]->prox = NULL;

    // 6. Corrige o ponteiro para o inicio da lista.
    *perif = vetor[0];
    //void imprimirLista(*perif);

    // 7. Deleta o vetor auxiliar.
    free(vetor);
}

void imprimirLista(Aux_Perif** lista){
	Aux_Perif* aux = *lista;
	while (aux != NULL){
		std::cout << "DISPOSITIVO: ";
		std::cout << aux->info.dispositivo << std::endl;
		std::cout << "PRIORIDADE: ";
		std::cout << aux->info.prioridade << std::endl;
		aux = aux->prox;
	}
}

void inserirFinal(Aux_Perif** perifericos, Perif barramento){
	Aux_Perif* novo = (Aux_Perif*)new Aux_Perif;
	novo->info = barramento;
	Aux_Perif* aux = *perifericos;
	Aux_Perif* anterior = NULL;
	if (*perifericos == NULL){
		*perifericos = novo;
	}else{
		while(aux != NULL){
			anterior = aux;
			aux = aux->prox;
		}
		anterior->prox = novo;
	}

}
\end{lstlisting}

\chapter{Análise dos Resultados}
\chapter{Conclusão}
\chapter{Material Utilizado}
\chapter{Descrição e Organização do Projeto e Executar}

\end{document}